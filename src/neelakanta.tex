\documentclass[a4paper]{report}
\usepackage[utf8]{inputenc}
\usepackage{amsmath}
\usepackage{amssymb}
\usepackage{amsfonts}
\usepackage{amsthm}
\usepackage{graphicx}
\usepackage[dvipsnames]{xcolor}
\usepackage{tcolorbox}
\usepackage[pdfpagelabels]{hyperref}
\usepackage{caption}
\usepackage{subcaption}
\usepackage{arcs}
\usepackage{xwatermark}
\usepackage{array}
\hypersetup{pdftex}
\pagecolor{white}
\color{black}

\hypersetup{
    colorlinks=true,
    linkcolor=blue,
    filecolor=magenta,      
    urlcolor=OliveGreen,
}

\usepackage{etoolbox}

\makeatletter
\providecommand\@gobblethree[3]{}
\patchcmd{\over@under@arc}
 {\@gobbletwo}
 {\@gobblethree}
 {}{}
\makeatother

\graphicspath{ {figures/} }

\newcommand{\mkfig}[5]{
  \begin{figure}[#1]
    \centering
    \includegraphics[width=#3in]{/home/achyut/latex/#4}
    \caption{\centering{#5}}
    \label{fig.#2}
  \end{figure}
}

\newcommand{\mksubfig}[5]{
  \begin{subfigure}[#1]{#3\textwidth} 
    \centering
    \includegraphics[width=\textwidth]{/home/achyut/latex/#4}
    \caption{\centering{#5}}
    \label{fig.#2}
  \end{subfigure}
}

\newtheorem{problem}{Problem}
\newtheorem{example}{Example}
\newtheorem{solution}{Solution}

\renewcommand*{\proofname}{Sol}


\title{Kepler's Laws}
\author{Achyut Bharadwaj (\href{mailto:achyut.22068@gear.ac.in}{achyut.22068@gear.ac.in})\\ 
  Daksh Anand (\href{mailto:daksh.22016@gear.ac.in}{daksh.22016@gear.ac.in})\\
  Aryan GV (\href{mailto:aryan.22157@gear.ac.in}{aryan.22157@gear.ac.in})\\ 
Medhajit Deb (\href{mailto:medhajit.22004@gear.ac.in}{medhajit.22004@gear.ac.in})}
\date{\today}

\setcounter{tocdepth}{4}
\setcounter{secnumdepth}{4}

\begin{document}
\maketitle
\tableofcontents
\pagebreak
\listoffigures
\noindent

\chapter{Some basic concepts needed for Kepler's laws}
\section{Angular displacement and velocity}
\subsection{Angular displacement}
Just as displacement is defined as the change in position of an object, angular displacement is defined as
the angle swept by an object travelling in a circular path.\\\\
Let's look at an example. Say, an object travels from point $A$ to point $B$. In this process, say it 
covers an angle of $\theta$ (See figure \ref{fig.4}).
\mkfig{h!}{4}{2}{presentation.4.png}{An example} \\
Here, arc \(\overarc{$AB$}\) is the tangential displacement and $\theta$ is the angular displacement.
In this case, $$\text{\(\overarc{$AB$}\)}=r\theta,$$ where $r$ is the radius of the circular path.
Hence, the relationship between angular displacement and tangential displacement is: $$s=r\theta,$$
where $s$ is the tangential displacement and $\theta$ is the angular displacement.

\subsection{Angular velocity}
Just as tangential velocity is displacement per unit time, angular velocity is the angle swept per unit time.
So, the average angular velocity can be written as: $$\omega_{avg}=\frac{\Delta \theta}{\Delta t},$$
where $\omega$ is the angular velocity, $\theta$ the angle swept and $t$ the time taken. \\\\
Now, as the change in time approaches 0, $\frac{\Delta \omega}{\Delta t}$ will be the instantaneous angular
velocity. Angular velocity as change in time approaches 0(instantaneous angular velocity) can be 
represented as: $$\omega=\frac{d\theta}{dt}.$$ Note that the $d$ before $\theta$ and $t$ is just a way to
represent the angular velocity as the change in time approaches 0. In general, $d$ is put before a variable to 
represent extremely small quantities.\\\\
The relationship between angular velocity and tangential velocity is: $$v=r\omega,$$
where $\omega$ is the angular velocity, $v$ is the tangential velocity and $r$ is the radius of the circular path.
We can also write the relationship as: $$\omega=\frac{v}{r}.$$

\section{Centripetal force}
Centripetal force is the force that acts on a body travelling in a circular path, towards the center of the 
circle. The numerical value of centripetal force is given by: $$F_c=\frac{mv^2}{r},$$
where $F_c$ is the centripetal force, $m$ is the mass of the body, and $v$ is the tangential velocity of the 
body.\\\\
Now, let's say that a person is twirling a ball of mass $m$ attached to a light and inextensible string of length
$r$.
Let us draw all the actual forces acting on it(ignore gravity).
\mkfig{h!}{5}{2}{presentation.5.png}{Forces acting on mass} \\\\
The only apparent force that is present is the tension due to the string $(T)$(See figure \ref{fig.5}).
But, we know that there is a centripetal force acting on any body going around in a circular path.
So, we can conclude that in this case, the tension due to the string is the cause of the centripetal force.
Hence, $T=F_c$.\\\\
From this, we can generically conclude that the net external force acting on a body travelling in a circular 
path, towards the center of the circular path is equal to the centripetal force. \\\\
Now, let us apply this to something different: Gravity.

\section{Orbital velocity} \label{sec.2}
Let's start with a few problems.
  \begin{tcolorbox}
    \example{A person is twirling a ball of mass $m$ attached to a light, inextensible string of 
    length $r$ in space. Given that the tension in the string is $T$ N, find the tangential velocity.}
  \end{tcolorbox}
    \proof{
    We have already seen that the net external force acting towards the center of the circular path is equal to
    the centripetal force. Hence, in this case: $$F_c=\frac{mv^2}{r}=T.$$
  Hence, $$v=\sqrt{\frac{Tr}{m}}$$}
    \begin{tcolorbox}
      \example{A satellite of mass $m$ is revolving at a distance of $r$ from the center of a planet
      of mass $M$. Find the satellite's tangential velocity.}
    \end{tcolorbox}
    \proof{
    This problem is just like the previous one. The only difference is that instead of tension, there is a force
    of Gravity acting on the satellite. Hence, $$F_c=\frac{mv^2}{r}=\frac{GMm}{r^2}.$$
  So, $$v=\sqrt{\frac{GM}{r}}.$$}
If you could do the second problem on your own, you have figured out what this section is all about!\\

\section{Summary}
\begin{itemize}
  \item The relationship between angular velocity and tangential velocity is given by: 
    $$\omega=\frac{v}{r},$$ where $\omega$ is the angular velocity, $v$ the tangential velocity and $r$ the
    radius of the circular path.
  \item Centripetal Force is given by: $$F_c=\frac{mv^2}{r},$$ where $F_c$ is the centripetal force, 
    $m$ the mass of the object and $v$ the tangential velocity.
  \item The Orbital velocity of a satellite of mass $m$at a distance $r$ from the center of a planet of mass
    $M$ is given by: $$v_{\text{orbital}}=\sqrt{\frac{GM}{r}}.$$
\end{itemize}



\chapter{Kepler's Laws}
\section{Kepler's first law}
\subsection{Law statement}
Kepler's first law states that all planets go around the Sun in an elliptical orbit, and that the Sun is 
at one of the foci of the orbit.\\\\
But, to understand this, we must first understand what an ellipse is.
\subsection{Definition of Ellipse}
Let us first plot two points on a graph, and call it $F_1$ and $F_2$. Now, find a point which is say, 
a distance $r_1$ away from $F_1$, and a distance of $r_2$ away from $F_2$. Can you also find another point that
is a distance $s_1$ away from $F_1$ and $s_2$ away from $F_2$, where $$r_1+r_2=s_1+s_2?$$
Now, if you plot infinite points such that $$r_1+r_2=s_1+s_2=p_1+p_2\cdots,$$ where $r_1, s_1, p_1, \dots$ are the
distances of the points from $F_1$ and $r_2,s_2,p_2,\dots$ are the distances from $F_2$, what will you get? 
For the answer, see figure \ref{fig.1}.
\mkfig{h!}{1}{2}{presentation.1.png}{An ellipse} \\\\
You will get an ellipse! So, An ellipse is a plane curve surrounding two focal points, such that 
for all points on the curve the sum of the distances to the focal points is constant, 
just like how all the points in a circle is equidistant from its center.\\\\
In fact both circle and ellipse are conic sections that is a curve obtained from the intersection of a 
cone and a plane, along with parabola and hyperbola. \\\\
Now, it's easier to understand what Kepler's first law means. There is a proof for Kepler's first law, but
we are not going to prove it here.\\\\
Let's move on to the second law.

\section{Kepler's second law}
\subsection{Law Statement}
Kepler's second law states that a planet moving around the sun sweeps across equal areas in equal amounts of time.
\mkfig{h!}{2}{2}{presentation.2.png}{Kepler's second law} \\

\subsection{A relation between distance from planet and velocity}\label{sec.1}
Let's examine this closely. Just like velocity is defined as the displacement per unit time, we can define 
areal velocity as the area swept per unit time. We know that $$v_{\text{avg}}=\frac{\Delta s}{\Delta t},$$
where $\Delta s$ is the change in displacement and $\Delta t$ is the change in time. \\\\
So, we can say: 
\begin{align} \label{eq.1}
  \text{Areal velocity}_\text{avg} = \frac{\Delta A}{\Delta t},
\end{align}
where $A$ is the area.
What happens when $\Delta t$ approaches $0$? As $\Delta t$ approaches 0, the right hand side of equation \ref{eq.1}
becomes the instantaneous areal velocity instead of the average areal velocity. \\\\
So, another way to represent equation \ref{eq.1} when $\Delta t$ approaches 0, is: 
\begin{align} \label{eq.2}
  v_A=\frac{dA}{dt},
\end{align}
where $v_A$ is the instantaneous areal velocity. Now, there is no need to get scared of this. It is just a way
to represent the areal velocity as the change in time approaches 0. \\\\
Now, let us take a small portion of the ellipse, where the angle made by the line segments($l_1,l_2$) 
which extend till the periphery of the ellipse (see figure \ref{fig.3})is $d\theta$ (Note that the $d$ before the
$\theta$ represents the fact that $d\theta$ is very small). \\
\mkfig{h!}{3}{2}{presentation.3.png}{Section of ellipse} \\
Since the angle is very small, the length of the $l_1,l_2$ are equal. Hence, we can treat the section of the 
ellipse as the section of a circle with center $F_1$ and radius $r$, where 
$$r=\text{ length of }l_1=\text{ length of }l_2.$$
Now, what is the area of this section? Well, it is simply the area of the section of a circle with radius
$r$ making an angle of $d\theta$. \\
\\
In general, how do we find the area of a section of a circle given that the radius is say, $s$, and the 
angle it makes at the center is say, $\alpha^{\circ}$? \\\\
The area will be: $$(\pi s^2)\times \frac{\alpha}{360}.$$
But, now let us say that the angle is $\beta$ radians. Radians is the SI unit for angles. $2\pi$ radians is 
equal to $360^{\circ}.$ Hence, if the angle is $\beta$ in radians, the area will be: 
$$\frac{\beta}{2\pi}\pi s^2.$$
\\
We can apply this to the section of our ellipse, which we are treating as a section of a circle too. 
We know that the radius is $r$, and angle $d\theta$. So, the area($dA$) of the section can be written as:
\begin{align}\label{eq.3}
  dA=\frac{d\theta}{2\pi}\pi r^2=r^2\frac{d\theta}{2}.
\end{align}
\\
Plugging this into equation \ref{eq.2}, we get: 
$$v_A=\frac{r^2\frac{d\theta}{2}}{dt}=\frac{r^2}{2}\frac{d\theta}{dt}.$$
Since we know that $\omega=\frac{d\theta}{dt}$, we can rewrite the above as: $$v_A=\omega\frac{r^2}{2}.$$
Since we know that $\omega=\frac{v}{r}$, we can rewrite the equation as:
$$v_A=\frac{r^2}{2}\frac{v}{r}=\frac{vr}{2}.$$
Since $v_A$ is constant, we can write:
\begin{align*}
  &\frac{vr}{2}=k\\
  \Rightarrow &\boxed{vr=2k=c},
\end{align*}
where $k$ and $c$ are constants, and $c=2k.$
Hence, $v$ is inversely proportional to $r$, where $r$ is the distance between the planet and the Sun.

\section{Kepler's third law}
\subsection{Law Statement}
Kepler's third law states that the cube of the major axis of the orbit of a planet around the sun is directly 
proportional to the square of time it takes to move around the sun. The law can be expressed as: 
\begin{align*}
  &T^2\propto a^3\\
  \Rightarrow &T^2 = k\cdot a^3,
\end{align*}
Where $a =$ major axis of the planet's orbit around the sun, $T =$ time period of the planet (around the sun)
and $k$ is a constant.\\
\subsection{A little more about ellipses}
Now, what is the major axis? It is the distance between the center of an ellipse and farthest point from the 
center of the ellipse(See figure \ref{fig.6}).
\mkfig{h!}{6}{2}{presentation.6.png}{Major axis of ellipse} \\
In figure \ref{fig.6}, $a=\overline{OD}$ is the major axis. \\

\subsection{Derivation for circular orbits}
Let's now try to derive the law(for circular orbits only).
First, what is time period mathematically, for circular orbits? Time period is the time an object takes to 
complete a full circle. In other words, it is the time takes for the object to travel a 
distance of $2\pi r$ units(Circumference of a circle).\\\\
Now, consider a planet travelling at a tangential velocity $v$ in a circular path with radius $r$ around the Sun. 
The time period will be the time taken to travel a distance of $2\pi r$. Hence, we can say: 
\begin{align}\label{eq.4}
T=\frac{2\pi r}{v}.
\end{align}
Since we know that the tangential velocity of an object in orbit at a distance $R$ from the center of a planet 
of mass $M$ is 
given by $$v_{\text{tangential}}=\sqrt{\frac{GM}{R}},$$
we can substitute $$v=\sqrt{\frac{GM}{r}},$$
where $M$ is the mass of the Sun, into equation \ref{eq.4}. Doing so, we get: 
\begin{equation} \label{eq.5}
  \begin{split}
    T &= \frac{2\pi r}{\sqrt{\frac{GM}{r}}} \\
    &= \frac{2\pi r\sqrt{r}}{\sqrt{GM}}\\
    &= 2\pi\frac{r^3}{GM}
  \end{split}
\end{equation}
Now, squaring both sides of equation \ref{eq.5}, we get: $$T^2=4\pi^2\frac{r^3}{GM}=\frac{4\pi^2}{GM}r^3.$$
Since $\frac{4\pi^2}{GM}$ is a constant, we let $k=\frac{4\pi^2}{GM}$. Hence, 
$$T^2=kr^3,$$ or $$\boxed{T^2\propto r^3}.$$



\section{Summary}
\begin{itemize}
  \item Johannes Kepler was a 16th century astronomer who established three laws which govern the motion 
    of planets (around the sun). These are known as Kepler’s law of planetary motion. The same laws also 
    describe the motion of satellites (like the moon) around plants (like the earth).

  \item Though Kepler gave the laws of planetary motion, he could not give a theory to explain the motion of  
    planets. It was Newton who showed that the cause of the motion of planets is the gravitational force which 
    the sun exerts on them. In fact, Newton used the Kepler’s third law of planetary motion to develop the law  
    of universal gravitation.
  \item[]
    \paragraph*{Kepler’s first law}
    The planets move in elliptical orbits around the sun, with the sun at one of the two 
    foci of the elliptical orbit. 

  \item[]
    \paragraph*{Kepler’s second law} 
    Each planet revolves around the sun in such a way that the line joining the planet to the sun sweeps over 
    equal areas in equal intervals of time.

  \item[]
    \paragraph*{Kepler’s third law} 
    The cube of the length of the major axis of the orbit of the planet around the sun is directly proportional to 
    the square of time it takes to move around the sun (Time period). The law can be expressed as: 
    $$T^2\propto a^3,$$ or $$T^2=k\cdot a^3,$$ where $k$ is a constant, $a =$ major axis of the orbit of the 
    planet around the sun and $T = $time period of the planet 
    (around the sun). 

\end{itemize}



\chapter{Problems}
In this chapter, we are going to be exploring some problems based on what has been covered in this presentation.
Note that all the problems have been framed and solved by at least one of the authors. Any problems that are not
framed by the authors or any solutions that aren't written by the authors will have their sources cited.

\section{Problems} \label{sec.4}
Let us start with some problems. The solutions to the problems can be found in section \ref{sec.3}.\\\\
For any clarifications based on the problems, contact one of the authors: 
\begin{itemize}
  \item Achyut Bharadwaj: \href{mailto:achyut.22068@gear.ac.in}{achyut.22068@gear.ac.in}
  \item Daksh Anand: \href{mailto:daksh.22016@gear.ac.in}{daksh.22016@gear.ac.in}
  \item Medhajit Deb: \href{mailto:medhajit.22004@gear.ac.in}{medhajit.22004@gear.ac.in}
  \item Aryan GV: \href{mailto:aryan.22157@gear.ac.in}{aryan.22157@gear.ac.in}
\end{itemize}

\subsection{Based on Chapter 1}
  \begin{tcolorbox}
    \problem{Given that the mass of the Earth is $M$ and the mass of a satellite orbiting the 
    Earth is $m$, find the tangential velocity of the satellite given that it is orbiting the Earth in a 
  circular path at a distance of $r$ from the center of the Earth.}
  \end{tcolorbox}
  \begin{tcolorbox}
    \problem{A geocentric satellite is orbiting the Earth in a circular path (A geocentric 
    satellite is a satellite that is fixed over a certain position on the Earth). Given that the mass of the 
    Earth is $M$ and the radius of the Earth is $R$, find:
    \begin{enumerate}
      \item[(a)] The tangential velocity of the satellite.
      \item[(b)] The height (from the surface) at which the satellite orbits the Earth,
    \end{enumerate}
  }
  \end{tcolorbox}
  \begin{tcolorbox}
    \problem{Given that the mass of the Earth is $M$, the radius of the Earth is $R$ and that the 
    duration of a day is $86400$ seconds, find the tangential velocity of an object on the surface of the Earth 
    at the equator.}
  \end{tcolorbox}

\subsection{Based on Chapter 2}
\setcounter{problem}{0}
  \begin{tcolorbox}
    \problem{A satellite of mass $m$ is orbiting a planet of mass $M$ in an elliptical orbit such 
    that it's velocity as a function of time between point $A$ and point $B$ (See figure \ref{fig.7})
    is given by $$v(t)=t^2+23t+2.$$ Given that the distance between the planet and the satellite at point 
    $A$ is $x$, and that $t=0$ when the satellite is at point $A$, find the distance between the planet and 
  the satellite ($r$) as a function of time.}
  \end{tcolorbox}
  \mkfig{h}{7}{3}{presentation.7.png}{The satellite orbiting the planet}
  \begin{tcolorbox}
    \problem{A satellite takes $T_1$ seconds to orbit the Earth in a circular orbit at a distance 
    $r_1$ from the center of the Earth. If another satellite is orbiting the Earth in a circular orbit at a 
    distance $r_2$ from the center of the Earth, find the time period of the satellite orbiting at the height 
  $r_2-R$, where $R$ is the radius of the Earth.}
  \end{tcolorbox}
  \begin{tcolorbox}
    \problem{A satellite takes $T_1$ seconds to orbit the Earth in a circular orbit at a distance 
    $r_1$ from the center of the Earth. If another satellite orbiting the Earth in a circular orbit takes $T_2$
    seconds to come back to the same position with respect to the Earth, find the height at which it orbits 
    assuming that the direction of revolution of both satellites are in the same direction as the rotation of 
    the Earth.
  (Take the duration of a day as $T$ seconds).}
  \end{tcolorbox}

\section{Solutions} \label{sec.3}
Here are the solutions to the problems from section \ref{sec.4}.\\\\
For any clarifications based on the solutions, contact one of the authors: 
\begin{itemize}
  \item Achyut Bharadwaj: \href{mailto:achyut.22068@gear.ac.in}{achyut.22068@gear.ac.in}
  \item Daksh Anand: \href{mailto:daksh.22016@gear.ac.in}{daksh.22016@gear.ac.in}
  \item Medhajit Deb: \href{mailto:medhajit.22004@gear.ac.in}{medhajit.22004@gear.ac.in}
  \item Aryan GV: \href{mailto:aryan.22157@gear.ac.in}{aryan.22157@gear.ac.in}
\end{itemize}

\subsection{Solutions to Problems based on chapter 1}
\begin{solution} \normalfont
We have already seen that $$v_{\text{orbital}}=\sqrt{\frac{GM}{r}},$$ where 
$v_{\text{orbital}}$ is the tangential velocity of orbital of a satellite around a planet, $M$ the mass of the 
planet and $r$ the radius of the orbit (Refer to section \ref{sec.2}). \\\\
Directly applying this to the problem, we get $$\boxed{v=\sqrt{\frac{GM}{r}}}.$$\\
\end{solution}
\begin{solution} \normalfont
The time period of an object going in a circular path with a tangential velocity say $v$
is given by: $$T=\frac{2\pi r}{v},$$ where $T$ is the time period, and $r$ the radius of the circular path.(Time
period is distance covered divided by velocity, and the distance in a circular path is $2\pi r$).\\\\
Now, let's come back to the problem.\\\\ 
Let $v$ be the tangential velocity, $T$ be the time period of the Earth and $r$ by the distance between the 
satellite and the center of the Earth.\\\\
\subparagraph*{Part (a)} 
The tangential velocity will be: 
\begin{align}\label{eq.7}
  {v=\sqrt{\frac{GM}{r}}}.
\end{align}
Applying the formula for time period, we get: 
\begin{equation}\label{eq.8}
  \begin{split}
    &T=\frac{2\pi r}{v}\\
    \Rightarrow & r=\frac{Tv}{2\pi}
  \end{split}
\end{equation}
Substituting $v=\sqrt{\frac{GM}{r}}$ from equation \ref{eq.7} into equation \ref{eq.8}, we get: 
\begin{equation}\label{eq.9}
\begin{split}
  &r=\frac{T\sqrt{\frac{GM}{r}}}{2\pi}\\
  \Rightarrow &r\sqrt{r}=\sqrt{r^3}={T\frac{\sqrt{GM}}{2\pi}}\\
  \Rightarrow & r=\sqrt[3]{\frac{T^2GM}{4\pi^2}}\\
  \Rightarrow & r-R=\text{height from surface}=\boxed{\sqrt[3]{\frac{T^2GM}{4\pi^2}}-R}.
\end{split}
\end{equation}
\\
\subparagraph*{Part (b)} 
Substituting the result we got for $r$ in equation \ref{eq.9}, into equation \ref{eq.7}, we get:
$$v=\boxed{\sqrt{\frac{GM}{\sqrt[3]{\frac{T^2GM}{4\pi^2}}-R}}}$$
\end{solution}

\subsection{Solutions to Problems based on chapter 2}
\setcounter{solution}{0}
  \begin{solution} \normalfont
    We have seen that the areal velocity of a satellite in an elliptical orbit is:
    $$v_A=\frac{vr}{2},$$where $v_A$ is the areal velocity, $v$ the tangential velocity at a point and $r$ the
    distance between the point and the planet(Refer to section \ref{sec.1}).\\\\
    Letting areal velocity be $v_A$ and applying the concept to this problem, we get: 
    \begin{align} \label{eq.6}
      v_A=\frac{v(t)r(t)}{2}
    \end{align}
    But, what is $v_A$? How do we find it?\\\\
    Since $v_A$ is constant by Kepler's third law, $v_A$ at $t=0$ will be the same as the $v_A$ in equation 
    \ref{eq.6}.
    We can easily find both $v(0)$ and $r(0)$. So, we can simply substitute $t=0$ in equation \ref{eq.6} to 
    find the areal velocity. \\\\
    Substituting $t=0$ in the function $v(t)$, we get: $$v(0)=0^2+23\times 0 +2 = 2.$$
    We are also given that the distance between point $A$ and the planet is $x$(Refer to figure \ref{fig.7}) and 
    that $t=0$ at point $A$. So, $r(0)=x.$ Hence, $$v_A=\frac{2x}{2}=x.$$\\
    Substituting $v_A=x$ in equation \ref{eq.6}, we get: 
    \begin{align*}
      &x=\frac{v(t)r(t)}{2}\\
      \Rightarrow &v(t)r(t)=2x\\
      \Rightarrow &r(t)=\frac{2x}{v(t)}=\boxed{\frac{2x}{t^2+23t+2}}.
    \end{align*}
  \end{solution}
\end{document}
